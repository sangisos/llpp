\documentclass[a4paper,11pt]{article}
\usepackage[T1]{fontenc}
\usepackage{fullpage}
\usepackage{listings}
\usepackage[hidelinks]{hyperref}
%\usepackage{latex/c/style}
\usepackage{pdfpages}
% According to the 
% Line plot example using external data fiels.
%
% Author: Claudio Favi
\usepackage{tikz}
\usetikzlibrary{plotmarks}
% The data files, written on the first run.
\begin{filecontents}{pthreads.data}
#Threads	Time [s]
1	2.566
2	1.607
3	1.657
4	2.130
5	2.458
6	2.700
8	3.090
16	3.880
\end{filecontents}

\begin{filecontents}{omp.data}
#Threads	Time [ms]
1	1.980
2	0.950
3	0.977
4	0.859
5	1.372
6	1.313
8	1.555
16	1.946
\end{filecontents}

\begin{filecontents}{seq.data}
#Threads	Time [ms]
1	1.808
1	1.808
1	1.808
1	1.808
1	1.808
1	1.808
1	1.808
1	1.808
\end{filecontents}

% 1.2.c
\renewcommand{\thesubsubsection}{\thesubsection.\alph{subsubsection}}

\title{\textbf{Low-Level Parallel Programming I (course 1DL550) \\
    Uppsala University -- Spring 2016 \\
    Report for Assignment 1 by Team 1
  }
}

\author{Alexander Andersson \and Jin Wenting \and Erik Österberg}

\date{\today}

\begin{document}

\maketitle
\section{Running}
To compile and run: Press F5 once this project is opened in Visual Studio 2013.
Versions: Inside file ped\_model.cpp (path ../libpedsim/src/), at line number 38,
choose between SEQ, OMP or PTHREAD for sequential, OpenMP or Pthreads
implementations respectively.
Machine used: please refer to 1.2 Question section question C.

\section{Questions}
\subsection{A. What kind of parallelism is exposed in the identified method?}
Data parallelism.
\subsection{B. How is the workload distributed across the threads?}
Evenly, as the threads get equal number of agents.
\subsection{C. Which number of thread gives you the best results? Why?}
Two thread, because two real core on the test setup (windows lab room 2315:8).
\subsection{E. How does your Pthreads implementation differ from the chosen OpenMP pragma?}
More code was written with Pthread than OpenMP because OpenMP (parallel for)
performs the thread allocation and work distribution automatically, in comparison to
Pthread.
\subsection{D. Which version (OpenMP, Pthreads) gives you better results? Why?}
OpenMP was fastest in the benchmark.
OpenMP have a lot of optimizations, often tailored for the platform it is running on.
Beneath is a plot of comparison between three implementations.

\begin{tikzpicture}[y=1.5cm, x=0.7cm,font=\sffamily]
 	%axis
	\draw (0,0) -- coordinate (x axis mid) (16,0);
    	\draw (0,0) -- coordinate (y axis mid) (0,5);
    	%ticks
    	\foreach \x in {0,...,16}
     		\draw (\x,1pt) -- (\x,-3pt)
			node[anchor=north] {\x};
    	\foreach \y in {0,0.5,...,5}
     		\draw (1pt,\y) -- (-3pt,\y) 
     			node[anchor=east] {\y}; 
	%labels      
	\node[below=0.8cm] at (x axis mid) {Threads};
	\node[rotate=90, above=0.8cm] at (y axis mid) {Time [s]};

	%plots
	\draw plot[mark=*, mark options={fill=red}] 
		file {pthreads.data};
	\draw plot[mark=triangle*, mark options={fill=blue} ] 
		file {omp.data};
	\draw plot[mark=square*, mark options={fill=green}]
		file {seq.data};

	%legend
	\begin{scope}[shift={(4,4)}] 
	\draw (0,0) -- 
		plot[red,mark=*, mark options={fill=red}] (0.25,0) -- (0.5,0) 
		node[right]{pthreads};
	\draw[yshift=\baselineskip] (0,0) -- 
		plot[blue,mark=triangle*, mark options={fill=blue}] (0.25,0) -- (0.5,0)
		node[right]{omp};
	\draw[yshift=2\baselineskip] (0,0) -- 
		plot[green,mark=square*, mark options={fill=green}] (0.25,0) -- (0.5,0)
		node[right]{seq};
	\end{scope}
\end{tikzpicture}

%\begin{table}
%	\begin{tabular}{llll}
%Threads & pthread & openmp & seq \\
%1 & 967 & N/A & 780 \\
%2 & 749 & N/A & N/A \\
%3 & 936 & N/A & N/A \\
%4 & 1232 & 265 & N/A \\
%5 & 1887 & N/A & N/A \\
%6 & 2091 & N/A & N/A \\
%7 & 2200 & N/A & N/A \\
%8 & 2684 & N/A & N/A
%	\end{tabular}
%\end{table}

\subsection{Workload}
We each worked quite equally on the assignment.

\clearpage

% --- [ References ] -----------------------------------------------------------

% Use IEEE-style references.
%\bibliographystyle{ieeetr}
%\bibliography{ref}

% --- [ Appendix ] -------------------------------------------------------------

%\clearpage

%\appendix
%\setcounter{secnumdepth}{0}

%\section{Appendix A}

\end{document}
